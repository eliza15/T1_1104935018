\documentclass[12pt,a4paper]{article}
\usepackage[utf8]{inputenc}
\usepackage[spanish]{babel}
\usepackage{amsmath}
\usepackage{amsfonts}
\usepackage{amssymb}
\usepackage{graphicx}
\usepackage{hyperref}
\usepackage{Sweave}
\begin{document}
\Sconcordance{concordance:PARTEB.tex:PARTEB.Rnw:%
1 8 1 1 0 51 1 1 2 3 0 1 2 34 0 1 1 2 0 1 1 3 0 1 3 1 1}

\author{Johanna}
\begin{center}
\textbf{EVALUACIÓN}
\end{center}
\texttt{Nombre: Johanna Rivera}\\
\texttt{Docente: Pablo Ordoñez}\\
\texttt{Módulo: VI A}\\
\texttt{FECHA:09-12-2015}\\

\begin{center}
\section*{PARTE B}
\end{center}
\textbf{¿Es aplicable la ingeniería de software cuando se elaboran webapps? Si es así, ¿cómo puede modificarse para que asimile las características únicas de éstas?}\\

Si son aplicables los webapps en la ingenieria de Software ya que se pueden utilizar estas herramientas para poder realizar la producción de sofware de manera colaborativa , incluyendo miembros  para que ayuden durante todo el proceso de producción.

\textbf{TITANIC}\\

\texttt{Descripción}\\

Este conjunto de datos proporciona información sobre el destino de los pasajeros en el primer viaje fatal del trasatlántico Titanic, que se resumen de acuerdo a la situación económica (clase), el sexo, la edad y la supervivencia.\\

\texttt{Uso}\\
Titanic\\

\texttt{Formato}\\

Una matriz dimensional de 4 resultante de cruzada tabular con 2201 observaciones sobre 4 variables. Las variables y sus niveles son los siguientes:
No 	Name 	Levels
1 	Class 	1st, 2nd, 3rd, Crew
2 	Sex 	Male, Female
3 	Age 	Child, Adult
4 	Survived 	No, Yes\\

\texttt{Detalles}\\

El hundimiento del Titanic es un evento famoso, y nuevos libros siguen siendo publicados sobre el tema. Muchos hechos,  de conocidas las proporciones de los pasajeros de primera clase a la política de "mujeres y niños primero ', y el hecho de que esa política no era un éxito completo en el ahorro de las mujeres y niños en la tercera clase se reflejan en la supervivencia tarifas de diversas clases de pasajeros.

Estos datos fueron recogidos originalmente por la Junta Británica de Comercio en su investigación del hundimiento. Tenga en cuenta que no hay un acuerdo completo entre las fuentes primarias como a las cifras exactas a bordo, rescatados, o perdidos.

Debido, en particular, a la película de gran éxito 'Titanic', los últimos años vieron un aumento en el interés público en el Titanic. \\

\texttt{Fuente}\\

Dawson, Robert J. MacG. (1995), The ‘Unusual Episode’ Data Revisited. Journal of Statistics Education, 3. http://www.amstat.org/publications/jse/\\
v3n3/datasets.dawson.html

The source provides a data set recording class, sex, age, and survival status for each person on board of the Titanic, and is based on data originally collected by the British Board of Trade and reprinted in:

British Board of Trade (1990), Report on the Loss of the ‘Titanic’ (S.S.). British Board of Trade Inquiry Report (reprint). Gloucester, UK: Allan Sutton Publishing. 
\begin{Schunk}
\begin{Soutput}
[1] "TABLA DEL DATASET TITANIC"
\end{Soutput}
\begin{Soutput}
    X Class    Sex   Age Survived Freq
1   1   1st   Male Child       No    0
2   2   2nd   Male Child       No    0
3   3   3rd   Male Child       No   35
4   4  Crew   Male Child       No    0
5   5   1st Female Child       No    0
6   6   2nd Female Child       No    0
7   7   3rd Female Child       No   17
8   8  Crew Female Child       No    0
9   9   1st   Male Adult       No  118
10 10   2nd   Male Adult       No  154
11 11   3rd   Male Adult       No  387
12 12  Crew   Male Adult       No  670
13 13   1st Female Adult       No    4
14 14   2nd Female Adult       No   13
15 15   3rd Female Adult       No   89
16 16  Crew Female Adult       No    3
17 17   1st   Male Child      Yes    5
18 18   2nd   Male Child      Yes   11
19 19   3rd   Male Child      Yes   13
20 20  Crew   Male Child      Yes    0
21 21   1st Female Child      Yes    1
22 22   2nd Female Child      Yes   13
23 23   3rd Female Child      Yes   14
24 24  Crew Female Child      Yes    0
25 25   1st   Male Adult      Yes   57
26 26   2nd   Male Adult      Yes   14
27 27   3rd   Male Adult      Yes   75
28 28  Crew   Male Adult      Yes  192
29 29   1st Female Adult      Yes  140
30 30   2nd Female Adult      Yes   80
31 31   3rd Female Adult      Yes   76
32 32  Crew Female Adult      Yes   20
\end{Soutput}
\begin{Soutput}
[1] "CUAL ES EL NUMERO TOTAL DE CASOS EN EL DATASET"
\end{Soutput}
\begin{Soutput}
[1] 2201
\end{Soutput}
\end{Schunk}

\end{document}
